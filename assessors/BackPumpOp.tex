\subsection{Back Pump Operation}

Flood control operations that discharge excess from an EAA basin to
Lake Okeechobee (commonly referred to as backpumping) are controlled by
the volume of runoff in the basin.  In general, backpumping is avoided
until runoff reaches a critical level.  The {\tt thresholdfrac}
package computes backpumped flow based on the following criteria:

\begin{equation}
	Q_{bp} = 0.0, \qquad Q_{runoff} \le Q_{thresh}
\end{equation}

\begin{equation}
	Q_{bp} = (Q_{runoff} - Q_{thresh}) * frac, \qquad Q_{runoff} > Q_{thresh}
\end{equation}

where:

\begin{center}
\begin{tabular}{rcl}
	$Q_{bp}$	&=& backpumped flow (cfs)					\\
	$Q_{runoff}$	&=& basin runoff (cfs)						\\
	$Q_{thresh}$	&=& threshold basin runoff (cfs)				\\
	$frac$		&=& fraction of basin runoff above threshold to be backpumped	\\
\end{tabular}
\end{center}

Backpumping operations are typically implemented by a Regional Manager
(see Chapter \ref{chapter:RegionalManager}) where the backpumping flow
computed by this assessor is monitored by an assessor monitor and used
to set a management constraint for flood control.  The decision tree
tests if the lake is above a specified backpump level \-- if it is,
set the management constraint to 0.0, otherwise, set the management
constraint to the backpump flow computed by this assessor.

