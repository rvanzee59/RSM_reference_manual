\subsection{Pulse Release}
The {\tt std} flood control assessor includes a {\tt pulse} option
to simulate the pulse releases from a lake through selected flood
control outlets (see Section \ref{fcassessor:std}).  The {\tt pulse}
option is used to control flood releases to environmentally sensitive
areas such as estuaries where salinity levels are important.  Studies
have shown that flow discharges into estuaries should simulate the
natural stormwater runoff which occurs in a pulsing discharge manner.
Discharges at the designated outlet follow a predefined flow
hydrograph for a specified number of days.

The {\tt pulse} option manages pulse releases from multiple outlets
(e.g., S77 to the Caloosahatchee Estuary and S308 to the St Lucie
Estuary).  The pulse release properties for each outlet are defined in
{\tt pulseRelease} special purpose assessor.  The {\tt pulse} option
refers to each {\tt pulseRelease} by specifying their respective
asmtID.

The {\tt pulseRelease} assessor supports multiple levels with flow
hydrographs that approximate the hydrologic response of a drainage
basin to storms of varying intensities.  This feature enables lake
managers to initiate a particular pulse level depending on the
severity of the flood control condition of the Lake.  The pulse
release levels are defined by multiple {\tt pulseHydrograph}
entries.

A pulse release can be initiated in two ways.  The standard option is
based only on lake stage.  A pulse release zone is associated with
each {\tt PulseHydrograph}.  The zone is defined by top and bottom
rulecurves and if the lake stage falls within the zone, a pulse
release is initiated.  For each day in the pulse release period, the
outlet release is determined by the pulse release hydrograph.  Once
initiated, the pulse release is completed, even if the lake stage falls
outside the zone.

The second option for initiating a pulse release relies on a wse
schedule defined by a regional manager.  The Water Supply and
Environmental (WSE) Schedule was developed to manage flood control
releases for Lake Okeechobee, considering a local and global climatic
influences.  This option uses a decision tree manager to assess the
condition of the lake. The decision tree manager traverses through a
collection of decision branches and until it reaches a terminal point
where a management constraint is defined for an mseNode.  The WSE
Decision Tree specifies pulse release at several terminal points.
These points are assigned to management constraints.  The wse {\tt
pulseHydrograph} package accesses the decision tree manager to
determine which (if any) pulse release should be initiated.  As with
the standard pulse release, once pulse release has been initiated,
outlet releases are set by the pulse hydrograph for the duration of
the pulse period.

