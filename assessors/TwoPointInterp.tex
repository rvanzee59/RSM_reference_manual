\subsubsection{{\tt TwoPointInterp} package}

Water levels in WCU's are commonly managed through regulation
schedules, where structure operations are based on the state of the
system and calendar date.  A regulation schedule is comprised of
``management zones'' that prescribe an operational response for a
defined set of hydrologic criteria (see Section
\ref{Section:DecisionTreeManagement}).  If the value returned by a
target monitor falls between the management zone's top and bottom
rulecurves, a management constraint (manCon) is set to a specified
value.  The {\tt TwoPointInterp} package is used to set the manCon to
a value based on the relative position of the target value in the
range defined by the two rulecurves.  The manCon values associated
with each rulecurve are combined by a weighting function:

\begin{align}
  weight &= 1.0,& target &\ge rulecurve_B \\
  weight &= \frac{target - rulecurve_C}{rulecurve_B - rulecurve_C},& rulecurve_B &> target > rulecurve_C \\
  weight &= 0.0, &target &\le rulecurve_C 
\end{align}

where 
\begin{center}
\begin{tabular}{rcl}
	$weight$	&=& weighting factor					\\
	$target$	&=& monitored time series value				\\
	$rulecurve_B$	&=& top value of the range				\\
	$rulecurve_C$	&=& bottom value of the range				\\
\end{tabular}
\end{center}

The weighted manCon value is computed by:

\begin{equation}
	value_{weighted} = weight * manCon_B + (1-weight) manCon_C
\end{equation}

where 
\begin{center}
\begin{tabular}{rcl}
	$manCon_B$	&=& managed constraint value for $rulecurve_B$	\\
	$manCon_C$	&=& managed constraint value for $rulecurve_D$	\\
\end{tabular}
\end{center}

