\subsection{Water Control Unit Status}

The status of a water control unit can be assessed using a {\tt
  WcuStatus} assessor.  In particular, {\tt WcuStatus} assessors are
used to compute the water supply deficit and flood control excess for
a water control unit.

The water supply deficit is defined as the volume of water required to
raise the water level up to a target deficit level.  The computed
deficit includes the net contribution of the HPM.  The flood control
excess is the volume of water required to lower the water level down
to a target flood control.  The computed excess includes net
contributions of the HPM, inlet inflows, unmanaged outlet releases,
outlet water supply releases, and boundary flows.  

The {\tt wcuDeficit} and {\tt wcuExcess} modes were designed to assess
the water supply and flood control status of basins and lakes.  The
{\tt canalExcess} mode was developed to designed to assess the flood
control status of particular type of water control unit \-- an EAA
canal.  The excess in an EAA canal is computed by executing the flood
control assessors assigned to the agricultural and stormwater
treatment areas, and reservoirs adjacent to the EAA canal.  Canal
excess is the summation of flood control releases through the
respective outlets to the EAA canal.  
