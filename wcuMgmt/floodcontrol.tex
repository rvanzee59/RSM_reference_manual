\section{Flood Control Assessors}\label{fcAssessor}

The flood control assessor quantifies the volume of excess in the
water control unit (WCU) and coordinates releases through managed
outlet structures to meet the WCU's needs as well as the regional
water supply and flood control requirements from upstream / downstream
WCU.  The {\tt default} flood control assessor package uses the
following procedure to assess WCU needs and coordinate its outlet
structure releases:

\begin{enumerate}
 \item Compute surplus flow for the water control unit.  Surplus flow
   is defined as the net inflow from WCU inlets, boundary conditions,
   unmanaged outlets, and water supply deliveries through WCU outlets.

 \item Compute stress from the WCU's hydrologic process module (HPM).

 \item Compute projected WCU storage, based on surplus flow and HPM
   stress.

 \item Compute WCU excess, defined as the volume of projected storage
   above the WCU's flood control level

 \item Set outlet releases to remove WCU excess and deliver water
   supply to meet downstream WCU needs.  Two options are available for
   distributing excess to multiple flood control outlets:

  \begin{enumerate}

    \item capacityweighted \-- distribution is weighted, based on the
      relative flood control capacity of the outlets.

    \item priorityorder \-- distribution is based on priority order
      assigned to the respective flood control outlets.

  \end{enumerate}

 \item Refine outlet releases.  Tests the downstream impact of the
   computed outlet flood control release. The degree to which
   downstream impacts are tested is controlled by the water control
   unit's fcRecursionLevel:

  \begin{enumerate}

     \item level 0 \-- do not check any downstream water control units
     \item level 1 \-- only check the flood control assessor
      immediately downstream of the water control unit
     \item level 2 \-- recursively check flood control assessors in
      downstream water control units
  \end{enumerate}
\end{enumerate}

Flood control releases at the outlet structures are limited by
capacity and management conveyance constraints.  If the downstream
water control unit is a lake with a specified fullLevel, the flood
control release is further constrained by the downstream storage
capacity.

The {\tt default} package's refine outlet flood control procedure
works by executing downstream flood control assessors as dictated by
the fcRecursionLevel and applying conveyance test to their respective
flood control outlets.  If the test fails, the flood control release
is reduced by 10\% and the downstream flood control assessor(s) are
re-executed.  This cycle of downstream flood control assessment, test,
and release reduction is repeated until the conveyance test passes or
the flood control release is reduced to zero.  The conveyance test is
specified as an {\tt dsConveyTest} for the downstream flood control
assessor (see Table \ref{WBodyFCProp}).  The following conveyance
tests are available:

\begin{enumerate}
 \item HeadTailTest \-- the estimated tail water head is compared with
   tail water head limit specified in MSE Network.  If estimated tail
   water head is higher, the test fails.

 \item LocRunoffLim \-- flood control release is limited by the local
   runoff in the downstream WCU.  Flood control release is adjusted,
   and test always passes.

 \item CapacityTest \-- flood control release is compared to the flood
   control capacity based on estimated head water and tail water heads
   at the outlet.  If release exceeds test capacity, test fails.

 \item NoFCInletRule \-- no test is applied.  Test always passes.

\end{enumerate}

The default flood control assessor methods can be overloaded by
specifying the {\tt package} attribute and specifying additional
parameters through nested elements (if needed).  Flood control
packages for WCU assessors are described and their input
specifications are defined in the following subsections.

