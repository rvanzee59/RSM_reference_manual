\subsection{{\tt lkissQTW} package }

Flood control assessment for the combined BC Pool in the Kissimmee
River can be simulated using the {\tt lkissQTW} package.  Pool BC is
simulated as single level pool, when in reality, the restored
Kissimmee River meanders and water level changes significantly from
north to south, especially during high flow events.  The headwater
level in Pool BC (i.e., the tailwater head for the inlet structure) is
approximated using a flow versus structure tailwater head lookup
table.  This approximation is needed to provide more realistic
tailwater head estimate for the inlet capacity and gate opening
criteria computations.

The flood control needs in the {\tt lkissQTW} package are assessed
using the methods developed for the {\tt default} fcAsesessor:

\begin{enumerate}
 \item Compute surplus flow in the WCU.  This includes inflows from
   inlet nodes and boundary conditions and outflows from unmanaged and
   water supply releases from outlet nodes.

 \item Compute HPM stress from the WCU's hydrologic process module.

 \item Use surplus flow and HPM stress to compute projected WCU
   storage.

 \item Compute excess volume in the WCU, defined as the portion of
   projected storage above the WCU's flood control level

 \item Set outlet flood control releases.  The {\tt lkissQTW} package
   uses the same release methods, with the same options, as the {\tt
   default} fcAssessor package.

 \item Refine outlet flood control releases.  The {\tt lkissQTW}
   package uses the same refine methods, with the same options, as the
   {\tt default} fcAssessor package.  However, the water level at the
   upstream end of the WCU is updated using the flow versus inlet
   tailwater head lookup table.

\end{enumerate}

