\subsection{{\tt std} package}

Standard water supply assessment for lakes and basins can be
simulated using the {\tt std} package.  The standard assessor can be
used to simulate the water supply needs for a variety of landscapes,
using different management options.  

The {\tt std} package uses the following procedures to compute water
supply requirement for the lake or basin and quantify the unmet water
supply requirements to be met by regional deliveries through the
inlets:

\begin{enumerate}
 \item Compute local water supply requirement.  The default method
   defines water supply requirement as the inflow needed to raise the
   water level up to the specified maintenance level.  This includes
   net contribution from the HPM.  The local water supply requirement
   can also be based on a preprocessed time series using the {\tt
   -predef} package modifier (e.g. {\tt package="std-predef"}).

 \item Sum boundary condition flows, outflow from unmanaged outlets,
   and inflow from unmanaged inlets.

 \item Compute unmet water supply requirement.

 \item Compute optional environmental inflow target \-- Inflow targets
   are typically defined for Stormwater Treatment Areas (STA's) and
   are designated as deliveries to an ``envTarg'' landscape.  The
   following options are available to compute the inflow target:

   \begin{enumerate}

    \item {\tt flow} - The user defines an outlet flow time series
      target for the STA outlet.  The inflow time target is the sum of
      the outlet time series plus the flow required to ``prime'' the
      STA, i.e., the volume of inflow needed to raise water level up
      to the {\tt fcLevel}.

    \item {\tt stage} - The inflow target is the inflow required to
      raise the water level up to the STA's {\tt fullLevel}.
   \end{enumerate}

 \item Set regional water supply requirements by distributing the
   unmet water supply demands to the water supply inlets. The
   following {\tt demandDist} options are available:

   \begin{enumerate}

   \item {\tt equal} - distributes unmet water supply requirement to
     the water supply inlets based on their relative available
     conveyance capacity.

   \item {\tt retReplace} - unmet water supply requirement is limited
     to the net reference evapotranspiration computed by the HPM.
     Unmet water supply requirement is distributed to the water supply
     inlets based on their relative available capacity or by the {\tt
     wsWeight} factor specified for the respective water supply
     inlets.
     
   \end{enumerate}
\end{enumerate}

A package modifier is used to designate the landscape type for the
computed water supply delivery.  This becomes important when the water
supply distribution from upstream sources is prioritized based on
landscape type.  A package modifier is formed by inserting a
recognized landscape type after {\tt std-} in the package
specification.  The water supply deliveries are recognized for the
following landscape types:

\begin{itemize}

  \item {\tt std-ag} \-- agricultural areas (default).

  \item {\tt std-env} \-- maintenance flows to environmentally
    sensitive lands (e.g., stormwater treatment areas, water
    conservation areas, wildlife management areas, etc).

  \item {\tt std-urban} \-- high to low density urban areas.

  \item {\tt std-tribe} \-- tribal lands.

  \item {\tt std-envTarg} \-- target flows to environmentally sensitive
  areas (e.g. estuaries).

\end{itemize}

