\section{Water Supply Assessors}\label{wsAssessor}

The water supply assessor quantifies the water supply requirement for
the WCU and sets the inlet water supply requirements for regional
water supply deliveries.  The {\tt default} water supply assessor uses
the following procedure to assess the regional water supply
requirements for a WCU:

\begin{enumerate}
 \item Compute local water supply requirement.  This is defined as
   the inflow required to offset the HPM contribution and raise the
   WCU's water level up to the maintenance level.

 \item Compute net regional water supply requirement.  This includes
   terms for outlet water supply requirements, unmanaged outlet flows,
   flood control inflows, and boundary conditions.

 \item Compute WCU deficit (local plus net regional water supply
   requirement)

 \item Distribute the WCU deficit by setting the water supply
   requirement on each water supply inlet.

\end{enumerate}

The default water supply assessor provides two options for
distributing the WCU deficit to the inlets:

\begin{enumerate}

 \item {\tt equal} \-- distribute water supply deficit to multiple
   water supply inlets based on their relative available conveyance
   capacity.

 \item {\tt priorityorder} \-- process the water supply inlets in
   priority order, fully utilizing each inlet until the deficit has
   been fully satisfied.  Water supply releases at inlets are limited
   by capacity and management constraints, and the available storage
   upstream (based on the upstream reserve level).

\end{enumerate}

The default water supply assessor methods can be overloaded by
specifying a ``package'' attribute and specifying additional
parameters through nested elements (if needed).  Water supply packages
are described in the following subsections.
