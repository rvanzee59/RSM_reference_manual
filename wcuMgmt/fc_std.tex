\subsection{{\tt std} package}\label{fcassessor:std}

Standard flood control assessment for lakes and basins can be
simulated using the {\tt std} package.  The standard assessor can be
used to simulate the water supply needs for a variety of landscapes,
using different management options.  

The {\tt std} package uses the following procedure to compute excess
flow in the WCU, allocate water supply releases to water supply
outlets and set flood control releases for the flood control outlets:

\begin{enumerate}

 \item Compute excess volume in the WCU.  Excess volume is defined as
   the portion of projected storage that exceeds the flood control
   level specified for the WCU.  The projected WCU storage includes
   flow terms from inlet structure, unmanaged outlet structures,
   boundary conditions and HPM stressors.

 \item Apply optional lag fraction based on excess versus lag fraction
   lookup table.

 \item Route water supply releases to water supply outlets.  The water
   supply allocation is based on the ratio of available supply to the
   total water supply requirement.  Available water supply is defined
   as the volume of projected storage that exceeds the storage at the
   {\tt resLevel}.  If demand exceeds available supply, an allocation
   fraction is applied to all water supply releases.  

 \item Compute optional pulse flood control release.  Test the outlet
   nodes that have been designated as pulse release nodes.  If they
   are active, set pulse release to current value in the pulse release
   hydrograph, subject to conveyance limitations.  

 \item Route remaining excess flow to the flood control outlets using
   one of the following {\tt excessDist} options:

 \begin{enumerate}
  
  \item {\tt priorityOrder} (default) \-- releases to flood control
    outlet structures are computed in priority order.  Projected head
    in the WCU is updated after each flood control release, based on
    the volume that remains in storage.  The flood control release for
    each outlet is constrained by volume of remaining excess flow,
    outlet capacity (physical and design), flood control management
    constraints and available storage downstream.

  \item {\tt fcWeight} \-- excess flow is apportioned to flood control
    outlets based on a specified weighting factor.  The flood control
    release for each outlet is constrained outlet capacity (physical
    and design), flood control management constraints and available
    storage downstream.

  \item {\tt fcWeightDRE} \-- excess flow is apportioned to flood control
    outlets based on a specified weighting factor.  The flood control
    release for each outlet is constrained outlet capacity (physical
    and design), flood control management constraints and available
    storage downstream.  If there is remaining excess flow, it is
    distributed to the outlets, based on their remaining capacity.

  \item {\tt rungeKutta} \-- The Runge\--Kutta approximation method is
    used to compute flood control releases from head dependent outlet
    structures.  

  \item {\tt dsArea} - excess flow is apportioned to flood control outlets
    based on the area of the receiving water body.  The flood control
    release for each outlet is constrained outlet capacity (physical
    and design), flood control management constraints and available
    storage downstream.

  \item {\tt capacityWeighted} \-- excess flow is apportioned to flood
    control outlets using on a weighting factor based on the relative
    capacity of the flood control outlets.  The flood control release
    for each outlet is constrained outlet capacity (physical and
    design), flood control management constraints and available
    storage downstream.

 \end{enumerate}

\end{enumerate}

The standard assessor for flood control simulates a number of
management options.  Selected outlets can be designated as pulse
release nodes by using the {\tt -pulse} package modifier.  Runoff
removal rates from the WCU can be controlled using the {\tt -lag} or
{\tt -lagdepth} package modifier (e.g. {\tt package="std-pulse"}, {\tt
  package="std-lag"}).

\textbf{\underline{Known Limitations}.}  

\begin {enumerate} 

 \item Does not support the recursive checking for downstream
 conveyance constraints as computed by the {\tt default}'s
 RefineOutletQFC() method.

 \item The {\tt std} flood control and water supply assessors can not
 be guaranteed to work with other assessor packages or types because
 other water supply assessors may set their inlet flows and other
 flood control assessors generally do not assume responsibility for
 setting outlet water supply releases.  For now, the {\tt std}
 assessors are designed to work with each other as well as the {\tt
 eaaCanal} assessors.

\end{enumerate}


