\section{LP Supervisors}

The runLP implementation uses a hybrid approach to manage the regional
system. North of Lake Okeechobee, an assessor coordinator routes water
through the Upper Chain of Lakes, Kissimmee, Caloosahatchee and St
Lucie river basins.  South of Lake Okeechobee, an LP supervisor routes
water through the canal network in the EAA.  The LP supervisor uses a
linear program (LP) solver to find an optimal set of structure
releases that meets water supply and flood control needs in the EAA,
subject to a series of basin storage and structure flow constraints.

The {\tt <glpk\_supervise>} element contains the specifications
required to build an independent LP model for the structures
supervised by LP.  The attributes used specify administrative details
such as the name of the LP model, input filenames, data formats, and
optimization options are listed in Table \ref {attrdefs}.

\begin{table}[!htb]
 \begin{center}
  \footnotesize
  \caption{Parameter definitions for {\tt glpk\_supervise} supervisor. }\label{attrdefs}
  \begin{tabular}{p{2.0cm}p{5.0cm}p{4.0cm}}                            \\[0.8ex]
   attribute  & Definition                    & typical value          \\
   \hline
   label      & management label              & "LP\_supervise"        \\
   modelName  & alternative label             &  "rsmbnLP"             \\ 
   modelFile  & model definition filename     & "input/mse/lp/lp.mod"  \\
   dataFile   & data specification filename   & "input/mse/lp/lp.dat"  \\
   dataFormat & data specification format     & "tabbed"               \\
   probFile   & optional output filename      & "none"                 \\
   solnFile   & optional solution filename    & "none"                 \\ 
   outputFile & optional output filename      & "none"                 \\
   method     & optimization procedure method & "simplex"              \\
   optimize   & optimization method           & "minimize"             \\
   presolve   & presolve option               & "off"                  \\
   msglevel   & message level                 & "1"                    \\
   timelimit  & time limit                    & "60"                   \\
   outfreq    & output frequency              & "1E6"                  \\
   days       & days                          & "0"                    \\ 
   hours      & hours                         & "0"                    \\
   minutes    & minutes                       & "0"                    \\
   \hline
  \end{tabular}
 \end{center}
\end{table}
\normalsize

The {\tt <glpk\_supervise>} element also defines the elements that
exchange data between the LP supervisor and the RSM.  The {\tt
  <varOut>} element takes the optimal structure flow computed by the
LP supervisor and assigns it to the corresponding watermover in the
RSM.  The {\tt <varIn>} element takes a state variable data from the
RSM and assigns it to selected parameters in the LP supervisor.

The LP model is defined in the file referenced by the modelFile
attribute.  This file contains definitions for:
\begin{enumerate}
 
 \item BASIN and STRUCTURE sets \-- collections of basin and structure
   members in the LP that are counterparts to respective waterbodies and
   watermovers in the RSM;

 \item general parameters \-- defines time step and unit conversions;

 \item basin and structure parameters \-- terms whose value is
   specified by RSM state variables through {\tt <varIn>} elements;
   and

 \item basin and structure decision variables \-- terms whose value
   is computed by the LP solver.

\end{enumerate}

The parameter and decision variable terms are used to form the linear
equations that define the objective function and a set of model
constraints.  These linear equations are defined in the file
referenced by the {\tt modelFile} attribute.

Each member of the BASIN set has the following time variant
parameters:

\begin{enumerate}
 
 \item storageInit \-- beginning of time step storage in the basin.

 \item basintarget1 \-- basin storage target \#1.  Basin storage
   deviations above and below target \# 1 storage are minimized.  Used
   to push basin water levels up or down towards a target.

 \item basintarget2 \-- basin storage target \#2.  Basin storage
   deviations below target \# 2 are minimized.  Used to maintain
   minimum water levels in the basin.

 \item basintarget3 \-- basin storage target \#3.  Basin storage
   deviations above target \# 3 are minimized.  Used to set the upper
   limit of basin storage capacity.

 \item wsDemand \-- water supply demand.  Water supply delivery
   deviations from the demand target are minimized.

 \item runoff \-- basin runoff.  Defines the boundary flow to each
   basin.

\end{enumerate}

The initial state of each basin in the BASIN set is established in the
LP model by initial storage, water supply demand and boundary flow.
Water levels in the basin are ``pushed'' up or down to minimize
deviations from specified storage targets. The LP supervisor pushes
basin water levels up or down by coordinating releases through the
basin's inlet and outlet structures.  Structure flow is limited by
physical and design capacities, and regional management objectives
expressed through constraints.

The deviations associated with each of the basin parameter targets are
weighted to reflect the priority of the respective target in the
objective function.  The following weights are static and are
specified for each member of the BASIN set:

\begin{enumerate}
 
 \item w\_target1Deficit \-- basin storage deviation below target \#1.

 \item w\_target1Excess \-- basin storage deviation above target \#1.

 \item w\_target2Deficit \-- basin storage deviations below target \#2.

 \item w\_target3Excess \-- basin storage deviations above target \#3.

 \item w\_demandDeficit \-- water supply delivery deviations from the
   demand target.

\end{enumerate}

Each member of the STRUCTURE set is defined by its upstream and
downstream basin and the following time variant parameters:

\begin{enumerate}
 
 \item maxFlow \-- maximum flow capacity of the structure.

 \item minflow \-- minimum flow target.  Deviations from this target
   are minimized.

 \item manCon \-- management constraint.  Sets the upper limit of
   allowable flow through the structure.

\end{enumerate}

Deviations from the minimum flow target are weighted globally for all
structures by single w\_minFlowDev parameter.  Each structure in the
STRUCTURE set is further defined by the following static parameters:
 
\begin{enumerate}
 
 \item name \-- structure name (usually the label of the corresponding watermover).

 \item designCap \-- structure design capacity.

 \item useManCon \-- switch that controls if the optional managed
   constraint parameter (manCon) is used.

 \item w\_flow \-- structure flow weight.  Used to penalize flow
   through selected structures.

\end{enumerate}


\subsection{Objective Function} \label{objectiveFunction}

The LP ``solves'' by minimizing its objective function, subject to
model constraints.  The objective function includes flow and deviation
terms for all the basin and structure members in the BASIN and
STRUCTURE sets:

\begin{equation}
 \begin{split}
  & \textrm { minimize objective } : \\
  & \textrm { sum \{ i in BASIN \} (}   \\
  & \quad \textrm { w\_target1Excess [i] * target1Excess [i] } + \\
  & \quad \textrm { w\_target1Deficit [i] * target1Deficit [i] } +  \\
  & \quad \textrm { w\_target2Deficit [i] * target2Deficit [i] } + \\
  & \quad \textrm { w\_demandDeficit [i]  * demandDeficit [i] } + \\
  & \quad \textrm { w\_target3Excess [i]  * target3Excess [i] )} + \\
  & \textrm { sum  \{ (i,j) in STRUCTURE \} } \textrm { ( w\_flow [i,j] * flow [i,j] ) }+ \\
  & \textrm { sum  \{ (i,j) in STRUCTURE \} } \textrm { ( w\_minFlowDev * minFlowDev [i,j] ) }
 \end{split}
\end{equation}

The weight factors in the objective function static parameters defined
by the user.  The flow and target deviation terms are decision
variables that are computed by the LP solver.  These decision
variables are also used to form the linear equations that define model
constraints.

The objective function minimizes the summation of:

\begin{enumerate}
 
 \item deviations from basin storage targets, 

 \item deviations from water supply demand targets, 

 \item structure flow penalties, and

 \item deviations from structure minimum flows.

\end{enumerate}

A target or penalty term can be dropped from the objective function by
setting their corresponding weight factor to 0.0.  Likewise, the
priority of a target or penalty term can be elevated by increasing its
weight factor relative to the weight factors to assigned to other
targets and penalty terms.

\subsection{Model Constraints} 

Model constraints are linear equations that use decision variable
terms.  The following decision variables are computed by the solver
for every member of the BASIN set:

\begin{enumerate}
 
 \item storageFinal \-- end of time step basin storage 
 \item target1Excess \-- volume of basin storage in above target \#1
 \item target1Deficit \-- volume of storage below target \#1
 \item target2Deficit \-- volume of storage below target \#2
 \item target3Excess \-- volume of storage above target \#3
 \item demandDeficit \-- volume of water supply deficit
 \item wsDemandActual \-- volume of water supply delivered

\end{enumerate}

All basin decision variables are restricted to values greater than or
equal to 0.0.

The following decision variables are computed by the solver for every
member of the STRUCTURE set:

\begin{enumerate}
 
 \item flow \-- volume of flow from upstream to downstream basin
 \item minFlowDev \-- deviation from minimum flow target

The flow decision variable is defined for every member of STRUCTURE
set and is restricted to values greater than or equal to 0.0.

\end{enumerate}

Model constraints fall into several broad categories: continuity,
basin constraints, flow constraints and canal specific constraints.
Each are entered into the LP solver as linear equations comprised of
parameter and decision variables.  The following sections describe
each category in more detail.


\subsubsection{Continuity constraint}

The conservation of mass constraint includes terms for beginning of
time step basin storage (storageInit), runoff, water supply delivery
(wsDemandActual), structure inflow and outflow:

\begin{equation}
 \begin{split}
   \textrm {subject to } & MassConservation \textrm { \{ i in BASIN \}} : \\
   \textrm {storageFinal [i]} & = \textrm { storageInit [i]} \\
                   & + \textrm {runoff [i] * unitFlow} \\
                   & - \textrm {wsDemandActual [i] * unitFlow} \\
                   & + \textrm {sum \{ (j,i) in STRUCTURE \} }  \textrm {flow [j,i]} \\ 
                   & - \textrm {sum \{ (i,j) in STRUCTURE \} }  \textrm {flow [i,j]} 
 \end{split}
\end{equation}

where the values for storageInit and runoff are set by {\tt <varIn> }
elements.  The \textrm {sum \{ (j,i) in STRUCTURE \} \{flow [i,j]\}} terms
sum structure inflows and outflows for the respective basins.  The
wsDemandActual term is based on the demand deviation from water supply
demand parameter:

\begin{equation}
 \begin{split}
  \textrm { subject to } DemandDeviation  & \textrm { \{ i in BASIN \} } : \\
  \textrm { wsDemandActual [i] } & =  \textrm { wsDemand [i] - demandDeficit [i] } 
 \end{split}
\end{equation}

The demandDeficit term is minimized through its inclusion in the
objective function.  

\subsubsection {Basin Constraints }

Basin water levels are managed by minimizing deviations from storage
targets.  Target \#1 is designed to minimize deviations above and
below the target:

\begin{equation}
 \begin{split}
  \textrm { subject to } Target1Deviation  & \textrm { \{ i in BASIN \} } : \\
  \textrm { storageFinal [i] } & = \textrm { basinTarget1 [i] + target1Excess [i] - target1Deficit [i] }
 \end{split}
\end{equation} 

where the basinTarget1 parameter is set by a {\tt <varIn> } element.
The target1Excess and target1Deficit terms are both weighted and
minimized through their inclusion in the objective function.  Target
\#2 is designed to minimize deviations below the target:

\begin{equation}
 \begin{split}
  \textrm { subject to } Target2Deviation  & \textrm { \{ i in BASIN \} } : \\
  \textrm { storageFinal [i] } & \ge \textrm { basinTarget2 [i] - target2Deficit [i] }
 \end{split}
\end{equation}

where the basinTarget2 parameter is set by a {\tt <varIn> } element.
The target2Deficit term is weighted and minimized through its
inclusion in the objective function.  Target \#3 is designed to
minimize deviations above the target:

\begin{equation}
 \begin{split}
  \textrm { subject to } Target3Deviation & \textrm { \{ i in BASIN \} } : \\
  \textrm { storageFinal [i] } & \le \textrm { basinTarget3 [i] + target3Excess [i] }
 \end{split}
\end{equation}

where the basinTarget3 parameter is set by a {\tt <varIn> } element.
The target3Excess term is weighted and minimized through its inclusion
in the objective function.

\subsubsection {Flow Constraints }

Structure flow is constrained by minimum flow, maximum flow and
management constraints.  The minimum flow constraint is implemented as
a target by minimizing the deviation between minflow and flow:

\begin{equation}
 \begin{split}
  \textrm { subject to } MinimumFlow & \textrm { \{ (i,j) in STRUCTURE \} } : \\
  \textrm { minFlowDev [i,j] } & \ge  \textrm { minflow [i,j] * unitFlow - flow [i,j] } 
 \end{split}
\end{equation}

where the minflow parameter is set by a {\tt <varIn> } element.  The
minFlowDev and flow terms are minimized through their inclusion in the
objective function.  The maximum flow constraint sets an upper bound
on structure flow:

\begin{equation}
 \begin{split}
    \textrm {subject to } FlowCapacity & \textrm { \{ (i,j) in STRUCTURE \}} : \\ 
    \textrm {flow [i,j]} & \le \textrm {maxFlow [i,j] * unitFlow} 
 \end{split}
\end{equation}

where the maxFlow parameter is set by a {\tt <varIn> } element.
Managed constraints are applied only if the useManCon value for the
respective structure is set to 1.  When applied, the managed
constraint sets an upper bound on structure flow:

\begin{equation}
 \begin{split}
    \textrm {subject to } & ManagedConstraint \textrm { \{ (i,j) in STRUCTURE \} } :  \\ 
    &\textrm {if useManCon [i,j]} \\ 
    &\textrm {then flow [i,j]} \le \textrm { manCon [i,j] * unitFlow } 
 \end{split}
\end{equation}

where the manCon parameter is set by a {\tt <varIn> } element and
useManCon value is set in the static model data block for structures.

\subsubsection {Canal Specific Constraints }

The constraints applied to the members of the BASIN and STRUCTURE sets
are generically applied to all the basins and structures in their
respective sets.  Sometimes it is desirable to apply constraints to
selected basins and structures.  For example, five basins discharge
their runoff into the North New River \-- Hillsboro Canal. If the
total runoff exceeds the outlet flow capacity of the canal or the
downstream storage capacity, it is desirable to prorate the discharge
uniformly from each of the five basins.  The following template is
used as a constraint for each basin connected to the North New River
\-- Hillsboro Canal:

\begin{equation}
 \begin{split}
  \textrm { subject to } &subbasinMin: \\
  &\textrm { SubbasinMin } = \textrm {  minflow['subbasin', 'nnrHills'] * unitFlow } \\  
  \textrm { subject to } &subbasinDev: \\
  &\textrm { SubbasinMin } = \textrm { flow['subbasin', 'nnrHills'] + SubbasinExcess } \\ 
  \textrm { subject to } &subbasinExcess: \\
  &\textrm { SubbasinExcess } =\textrm {  minflow['subbasin', 'nnrHills'] * ExcessRatioNNR }
 \end{split}
\end{equation}

where the name of the basin is substituted for \textrm {'subbasin''}
and \textrm {'Subbasin''}.  The $subbasinMin$ constraint sets \textrm
{SubbasinMin} equal to the minflow for a specific structure.  The
$subbasinDev$ constraint computes the flow deviation from \textrm
{SubbasinMin} (i.e., \textrm {SubbasinExcess}).  The $subbasinExcess$
constraint requires that \textrm {SubbasinExcess} be equivalent to a
fraction of \textrm {SubbasinMin} (\textrm {ExcessRatioNNR}).  Since
\textrm {ExcessRatioNNR} is the same for all the basins connected to
the North New River \-- Hillsboro Canal, the constraints will enforce uniform
prorating for all the basin runoff.  

The following decision variables were defined for canal specific
constraints applied to the North New River \-- Hillsboro and Upper
Miami Canals:

\begin{enumerate}
 \item SSDDExcess \-- excess runoff from South Shore Drainage District
   to North New River \-- Hillsboro Canal
 \item SSDDMin \-- minimum flow from South Shore Drainage District to
   North New River \-- Hillsboro Canal
 \item ClosterExcess \-- excess runoff from Closter Farms to North New
   River \-- Hillsboro Canal
 \item ClosterMin \-- minimum flow from Closter Farms to North New River \--
   Hillsboro Canal
 \item ESWCDExcess \-- excess runoff from East Shore Water Control
   District to North New River \-- Hillsboro Canal
 \item \-- ESWCDMin \-- minimum flow from East Shore Water Control
   District to North New River \-- Hillsboro Canal
 \item S5AExcess \-- excess runoff from S5A basin to North New River
   \-- Hillsboro Canal
 \item \-- S5AMin \-- minimum flow from S5A basin to North New River
   \-- Hillsboro Canal
 \item NNRHillsExcess \-- excess runoff from North New River \-- Hillsboro
   Basin to North New River \-- Hillsboro Canal
 \item NNRHillsMin \-- minimum flow from North New River \-- Hillsboro
   Basin to North New River \-- Hillsboro Canal
 \item MiamiBasinExcess \-- excess runoff from Miami Basin to Upper
   Miami Canal
 \item MiamiBasinMin \-- minimum flow from Miami Basin to Upper Miami
   Canal
 \item SSDDMiaExcess \-- excess runoff from South Shore Drainage
   District to Upper Miami Canal
 \item SSDDMiaMin \-- minimum flow from South Shore Drainage District
   to Upper Miami Canal
 \item SFCDExcess \-- excess runoff from South Florida Conservancy
   District to Upper Miami Canal
 \item SFCDMin \-- minimum flow from South Florida Conservancy
   District to Upper Miami Canal
 \item L1EExcess \-- excess runoff from L1E Basin to Upper Miami Canal
 \item L1EMin \-- minimum flow from L1E Basin to Upper Miami Canal
 \item ExcessRatioNNR \-- excess ratio applied to basin runoff
   discharged to North New River \-- Hillsboro Canal
 \item ExcessRatioMiami \-- excess ratio applied to basin runoff
   discharged to Upper Miami Canal
\end{enumerate}


\subsection{LP Model Data}

The initial condition for each basin is established for the LP by
initial basin storage, boundary flow (runoff), water supply demand and
target storage levels \#1, \#2 and \#3.  All these parameter values
are updated daily for each basin by their respective {\tt <varIn>}
elements. By convention, each member of the BASIN set is entered into
a data block with initial parameter values for each of the respective
basins.  Likewise, each member of the STRUCTURE set is entered into a
data block with the structure's upstream and downstream basin name and
initial parameter values for each of the respective structures.  This
data structure simplifies the way parameters are updated in the RSM.
Since the values from the corresponding {\tt <varIn>} entries
overwrite the initial parameter values in the data blocks, the actual
values for the initial basin and structure parameters are disregarded.
The parameter data blocks for BASIN and STRUCTURE sets are specified
in the file referenced by the {\tt dataFile} attribute.

The list of members in the BASIN set and the source of data for each
basin parameter are presented in Table \ref{TVbasinPara}.  Possible
sources of data for basin parameter values include:

\begin{enumerate}
 
 \item wbhead \-- beginning of time step head simulated for the
   respective basin by the HSE.

 \item rulecurve \-- schedule specified in the MSE.

 \item constant \-- value specified in {\tt <varIn>} element.

 \item external \-- pre-processed times series data stored in DSS file.

 \item bcflows \-- summation of boundary condition flows computed by
   an HSE assessor.

\end{enumerate}

The maxFlow, minflow and manCon parameter values for each structure
are updated daily by their respective {\tt <varIn>} elements.  The
list of members in the STRUCTURE set and the source for each parameter
are presented in Tables \ref {strParaTs_1} and \ref {strParaTs_2}.
Possible sources for structure parameters values include:

\begin{enumerate}
 
 \item wmcap \-- maximum flow capacity computed by the HSE.

 \item assessor (FC) \-- flood control requirement computed by an {\tt
   <AssessorCoordinator> } flood control assessor.

 \item assessor (WS) \-- water supply requirement computed by an {\tt
   <AssessorCoordinator> } water supply assessor.

 \item constant \-- value specified in {\tt <varIn>} element.

 \item external \-- pre-processed times series data stored in DSS
   file.

\end{enumerate}

The LP data file is also used to specify values for static basin and
structure parameters.  The static basin parameters used to weight
deviations from basin storage and water supply targets are listed in
Table \ref {StaticbasinPara}.  The static structure parameters used to
define structure name, design capacity, optional use of manCon's and
structure flow penalty are listed in Tables \ref{strParaConst_1}
and \ref{strParaConst_2}.

\subsection {Optimization strategies}

\subsubsection {Canal}

Canals in the EAA are assumed to provide no local storage, which means
inflow must equal outflow for each time step.  This is accomplished it
the LP by by placing a high penalty (100) on deviations above and
below target \#1 storage for each canal in the EAA.  Target \#1
storage is computed by the respective {\tt <wbmonitor>}, based on a
target water level that coincides with the canal's {\tt fcLevel} and
{\tt maintLevel}.

\subsubsection {Service Area with HPM}

The hydrology for a service area basin is simulated by a hydrologic
process module (HPM) in RSM.  Assessors quantify the water supply and
flood control requirements for the basin and distribute the
requirements to the basin's inlets and outlets.  The water supply
requirement for an inlet is retrieved through a monitor ({\tt
  <qpackmonitor comp="ws" attr="req">}) and entered as a minflow and
maxFlow constraint on the corresponding structure in the LP.  Setting
minflow and maxFlow to the same value ensures that the water supply
requirement will be met (subject to capacity and supply constraints),
and the LP will not divert additional flow to the basin for storage or
as a pass through to meet downstream needs.  Likewise, the flood
control requirement for an outlet is retrieved through a monitor {\tt
  <qpackmonitor comp="ws" attr="req">} and entered as a minflow and
maxFlow constraint on the corresponding structure in the LP.  Setting
the minflow and maxFlow to the same value ensures that the flood
control flood control needs of the basin are meet, subject to capacity
and storage constraints, and the LP will not attempt to use basin
storage or use the basin as a pass through to to meet water supply
needs elsewhere.

The maxFlow constraint is entered into the LP as an inequality,
thereby guaranteeing the flow will not exceed maxFlow.  The minflow
constraint is entered as a target and the deviation from the target is
minimized.  The penalty for missing the minflow target is universally
set to 10.

\subsubsection {Service Area \-- without HPM}

The hydrology for some service area basins is simulated by other
models (e.g., SFWMM, AFSIRS\--WATBAL, etc).  The water supply and
flood control requirements for the basin's inlets and outlets are
preprocessed and retrieved through a monitor ({\tt <dssmonitor>}) and
entered as a minflow and maxFlow constraint on the corresponding
structure in the LP.  As with service areas with HPM's, minflow and
maxFlow values for the inlets and outlets are set to ensure
requirements are met and the basin is not used as a reservoir or as a
pass through facility.  Unlike basins with HPM's, water supply
delivered to the basin do not have an HPM to remove the water from
the RSM model domain.  Service areas without HPM's use on a demandNode
to remove the water supply delivery from the RSM domain.  The minflow
and maxFlow value for the demandNode is set to the same value as the
water supply requirement as the basin inlets.

by a hydrologic process module (HPM) in RSM.  Assessors quantify the
water supply and flood control requirements for the basin and
distribute the requirements to the basin's inlets and outlets.  The
water supply requirement for an inlet is retrieved through a monitor
({\tt <qpackmonitor comp="ws" attr="req">}) and entered as a minflow
and maxFlow constraint on the corresponding structure in the LP.
Setting minflow and maxFlow to the same value ensures that the water
supply requirement will be met (subject to capacity and supply
constraints), and the LP will not divert additional flow to the basin
for storage or as a pass through to meet downstream needs.  Likewise,
the flood control requirement for an outlet is retrieved through a
monitor {\tt <qpackmonitor comp="ws" attr="req">} and entered as a
minflow and maxFlow constraint on the corresponding structure in the
LP.  Setting the minflow and maxFlow to the same value ensures that
the flood control flood control needs of the basin are meet, subject
to capacity and storage constraints, and the LP will not attempt to
use basin storage or use the basin as a pass through to to meet water
supply needs elsewhere.

The maxFlow constraint is entered into the LP as an inequality,
thereby guaranteeing the flow will not exceed maxFlow.  The minflow
constraint is entered as a target and the deviation from the target is
minimized.  The penalty for missing the minflow target is universally
set to 10.


\subsubsection {Stormwater Treatment Area}
 * Canal storage is constant, i.e., inflow equals outflow for each time step.  This enforced by  


\begin{table}[!htb]
 \begin{center}
  \footnotesize
  \caption{Example of the LP time variant basin parameters. }\label{TVbasinPara}
  \begin{tabular}{p{2.3cm}p{1.8cm}p{1.8cm}p{1.8cm}p{1.8cm}p{1.8cm}p{1.8cm}}         \\[0.8ex]
   basin          &storageInit &target1   &target2   &target3   &wsDemand &runoff   \\
  \hline
  lo              &wbhead      &rulecurve &rulecurve & 1000     & 0       & 0       \\
  L8              &wbhead      & 10       & 10       & 1000     & 0       & 0       \\
  C51W            &wbhead      & 10       & 10       & 1000     & 0       & 0       \\
  C51             &wbhead      & 10       & 10       & 1000     & 0       & 0       \\
  L10L12          &wbhead      & 9.5      & 9.5      & 1000     & 0       & 0       \\
  nnrHills        &wbhead      & 8        & 8        & 1000     & 0       & 0       \\
  umiami          &wbhead      & 8.5      & 8.5      & 1000     & 0       & 0       \\
  lmiami          &wbhead      & 8        & 8        & 1000     & 0       & 0       \\
  L5E             &wbhead      & 9        & 9        & 1000     & 0       & 0       \\
  L3Canal         &wbhead      & 8        & 8        & 1000     & 0       & 0       \\
  L8basin         &wbhead      & 8        & 8        & 1000     & 0       &external \\
  C51Basin        &wbhead      & 8        & 8        & 1000     & 0       &external \\
  C51WBasin       &wbhead      & 8        & 8        & 1000     & 0       &external \\
  S5ABasin        &wbhead      & 5        & -5       & 1000     & 0       & 0       \\
  ebwcd           &wbhead      & 5        & -5       & 1000     & 0       & 0       \\
  eswcd           &wbhead      & 5        & -5       & 1000     & 0       & 0       \\
  closter         &wbhead      & 5        & -5       & 1000     & 0       & 0       \\
  ssdd            &wbhead      & 5        & -5       & 1000     & 0       & 0       \\
  sfcd            &wbhead      & 5        & -5       & 1000     & 0       & 0       \\
  NnrHillsBasin   &wbhead      & 5        & -5       & 1000     & 0       & 0       \\
  miamiBasin      &wbhead      & 5        & -5       & 1000     & 0       & 0       \\
  rotenberger     &wbhead      &rulecurve &rulecurve & 1000     & 0       &bcflows  \\
  holeyland       &wbhead      & 13       &rulecurve & 1000     & 0       &bcflows  \\
  L1EBasin        &wbhead      & 25       & 10       & 1000     & 0       & 0       \\
  L2CBasin        &wbhead      & 25       & 10       & 1000     & 0       & 0       \\
  annexBasin      &wbhead      & 25       & 10       & 1000     & 0       & 0       \\
  ussugarUnit2    &wbhead      & 25       & 10       & 1000     & 0       & 0       \\
  bigCypRes       &wbhead      & 25       & 10       & 1000     & 0       & 0       \\
  STA1InflowCanal &wbhead      & 16.5     & 16.5     & 1000     & 0       & 0       \\
  sta1e           &wbhead      & 15.565   & 14.815   & 18.315   & 0       &bcflows  \\
  sta1w           &wbhead      & 11.06428 & 10.31428 & 14.31428 & 0       &bcflows  \\
  STA2InflowCanal &wbhead      & 10       & 10       & 1000     & 0       & 0       \\
  sta2            &wbhead      & 11.41    & 10.6     & 14.6     & 0       &bcflows  \\
  sta34sc         &wbhead      & 8        & 8        & 1000     & 0       & 0       \\
  sta34           &wbhead      & 10.72    & 9.97     & 13.97    & 0       &bcflows  \\
  sta5            &wbhead      & 13.83    & 12.63    & 16.63    & 0       &bcflows  \\
  sta6            &wbhead      & 14.07    & 13.28    & 17.28    & 0       &bcflows  \\
  wca1            &wbhead      & 20       & 7        & 1000     & 0       & 0       \\
  wca2a           &wbhead      & 20       & 7        & 1000     & 0       & 0       \\
  wca3a           &wbhead      & 20       & 7        & 1000     & 0       & 0       \\
  dummyLake       &wbhead      & 0        & 0        & 1000     & 0       & 0       \\
  \hline
  \end{tabular}
 \end{center}
\end{table}
\normalsize

\begin{table}[!htb]
 \begin{center}
  \footnotesize
  \caption{Example of the LP static basin parameter values. }\label{StaticbasinPara}
  \begin{tabular}{p{2.5cm}p{2.2cm}p{2.2cm}p{2.2cm}p{2.2cm}p{2.2cm}}                  \\[0.8ex]
   basin          &w\_target1Deficit &w\_target1Excess  &w\_target3Excess &w\_target2Deficit &w\_demandDeficit \\
  \hline 
  lo              &   0         &    0          &    0         &     0         &    0 \\
  L8              & 100         &  100          &    0         &     0         &    0 \\
  C51W            & 100         &  100          &    0         &     0         &    0 \\
  C51             & 100         &  100          &    0         &     0         &    0 \\
  L10L12          & 100         &  100          &    0         &     0         &    0 \\
  nnrHills        & 100         &  100          &    0         &     0         &    0 \\
  umiami          & 100         &  100          &    0         &     0         &    0 \\
  lmiami          & 100         &  100          &    0         &     0         &    0 \\
  L5E             & 100         &  100          &    0         &     0         &    0 \\
  L3Canal         & 100         &  100          &    0         &     0         &    0 \\
  L8basin         & 100         &  100          &    0         &     0         &    0 \\
  C51Basin        & 100         &  100          &    0         &     0         &    0 \\
  C51WBasin       & 100         &  100          &    0         &     0         &    0 \\
  S5ABasin        &   0         &    0          &    0         &     0         &    0 \\
  ebwcd           &   0         &    0          &    0         &     0         &    0 \\
  eswcd           &   0         &    0          &    0         &     0         &    0 \\
  closter         &   0         &    0          &    0         &     0         &    0 \\
  ssdd            &   0         &    0          &    0         &     0         &    0 \\
  sfcd            &   0         &    0          &    0         &     0         &    0 \\
  NnrHillsBasin   &   0         &    0          &    0         &     0         &    0 \\
  miamiBasin      &   0         &    0          &    0         &     0         &    0 \\
  rotenberger     &   0         &    1          &  100         &   100         &    0 \\
  holeyland       &   0         &    1          &  100         &   100         &    0 \\
  L1EBasin        &   0         &    0          &    0         &     0         &    0 \\
  L2CBasin        &   0         &    0          &    0         &     0         &    0 \\
  annexBasin      &   0         &    0.001      &    0         &     0         &    0 \\
  ussugarUnit2    &   0         &    0.001      &    0         &     0         &    0 \\
  bigCypRes       &   0         &    0          &    0         &     0         &    0 \\
  STA1InflowCanal & 100         &  100          &    0         &     0         &    0 \\
  sta1e           &   0         &    0.001      &  100         &   100         &    0 \\
  sta1w           &   0         &    0.001      &  100         &   100         &    0 \\
  STA2InflowCanal & 100         &  100          &    0         &     0         &    0 \\
  sta2            &   0         &    0.001      &  100         &   100         &    0 \\
  sta34sc         & 100         &  100          &    0         &     0         &    0 \\
  sta34           &   0         &    0.01       &  100         &   100         &    0 \\
  sta5            &   0         &    1          &  100         &   100         &    0 \\
  sta6            &   0         &    1          &  100         &   100         &    0 \\
  wca1            &   0         &    0          &    0         &     0         &    0 \\
  wca2a           &   0         &    0          &    0         &     0         &    0 \\
  wca3a           &   0         &    0          &    0         &     0         &    0 \\
  dummyLake       &   0         &    0          &    0         &     0         &    0 \\
  \hline
  \end{tabular}
 \end{center}
\end{table}
\normalsize

\begin{table}[!htb]
 \begin{center}
  \footnotesize
  \caption{Example of the LP time varying structure parameter values (Part 1). }\label{strParaTs_1}
  \begin{tabular}{p{2.5cm}p{2.0cm}p{2.0cm}p{2.0cm}}            \\[0.8ex]
  structure        &maxFlow         &minflow          &manCon  \\
  \hline
  C10A             &wmcap           &assessor (FC)    & 100000  \\
  S352             &wmcap           &assessor (FC)    & 100000  \\
  S351             &wmcap           &assessor (FC)    & 100000  \\
  S354             &wmcap           &assessor (FC)    & 100000  \\
  C10              &assessor (WS)   &assessor (WS)    & 100000  \\
  C12              &assessor (WS)   &assessor (WS)    & 100000  \\
  C12A             &assessor (WS)   &assessor (WS)    & 100000  \\
  C4A              &assessor (WS)   &assessor (WS)    & 100000  \\
  S236             &assessor (WS)   &assessor (WS)    & 100000  \\
  S5AE             &wmcap           & 0               & 100000  \\
  L8canal2basin    &assessor (WS)   &assessor (WS)    & 100000  \\
  CWPB2            &external        &external         & 100000  \\
  S155A            &wmcap           & 0               & 100000  \\
  C51Wcanal2basin  &assessor (WS)   &assessor (WS)    & 100000  \\
  S319             &wmcap           & 0               & assessor\\
  C51canal2basin   &assessor (WS)   &assessor (WS)    & 100000  \\
  S155             &wmcap           & 0               & 100000  \\
  S5A              &wmcap           & 0               & 100000  \\
  L10L12toS5ABasin &assessor (WS)   &assessor (WS)    & 100000  \\
  S2               &wmcap           & 0               &assessor \\
  G371             &wmcap           & 0               & 100000  \\
  nnr2nnrHills     &assessor (WS)   &assessor (WS)    & 100000  \\
  S6               &wmcap           & 0               & 100000  \\
  G370             &wmcap           & 0               & 100000  \\
  S3               &wmcap           & 0               &assessor \\
  G373             &wmcap           & 0               & 100000  \\
  mc2miaBasin      &assessor (WS)   &assessor (WS)    & 100000  \\
  G372             &wmcap           & 0               & 100000  \\
  G410             &wmcap           & 0               & 100000  \\
  L4toBigCypRes    &assessor (WS)   &assessor (WS)    & 100000  \\
  G507             &wmcap           & 0               & 100000  \\
  s8               &wmcap           &external         & 100000  \\
  L4gap            &wmcap           & 0               & 100000  \\
  S7               &wmcap           &external         & 100000  \\
  S150             &wmcap           &external         & 100000  \\
  G407             &wmcap           & 0               & 100000  \\
  G353             &wmcap           & 0               & 100000  \\
  \hline                                                                            
  \end{tabular}
 \end{center}
\end{table}
\normalsize

\begin{table}[!htb]
 \begin{center}
  \footnotesize
  \caption{Example of the LP time varying structure parameter values (Part 2). }\label{strParaTs_2}
  \begin{tabular}{p{2.8cm}p{2.0cm}p{2.0cm}p{2.0cm}}                \\[0.8ex]
  structure        &maxFlow           &minflow          &manCon    \\
  \hline
  basin2L8canal    &assessor (FC)     &assessor (FC)    & 100000   \\
  L8DemandNode     &assessor (WS)     &assessor (WS)    & 100000   \\
  basin2C51Wcanal  &assessor (FC)     &assessor (FC)    & 100000   \\
  C51WDemandNode   &assessor (WS)     &assessor (WS)    & 100000   \\
  basin2C51canal   &assessor (FC)     &assessor (FC)    & 100000   \\
  C51DemandNode    &assessor (WS)     &assessor (WS)    & 100000   \\
  S5ABasin2L10L12  &assessor (FC)     &assessor (FC)    & 100000   \\
  S5ABasin2hills   &assessor (FC)     &assessor (FC)    & 100000   \\
  C10bp            &assessor (FC)     &assessor (FC)    & 100000   \\
  ebps             &assessor (FC)     &assessor (FC)    & 100000   \\
  C12bp            &assessor (FC)     &assessor (FC)    & 100000   \\
  esps             &assessor (FC)     &assessor (FC)    & 100000   \\
  C12Abp           &assessor (FC)     &assessor (FC)    & 100000   \\
  cfps             &assessor (FC)     &assessor (FC)    & 100000   \\
  C4Abp            &assessor (FC)     &assessor (FC)    & 100000   \\
  sbpsrips         &assessor (FC)     &assessor (FC)    & 100000   \\
  sswestps         &assessor (FC)     &assessor (FC)    & 100000   \\
  S236bp           &assessor (FC)     &assessor (FC)    & 100000   \\
  sfd5eps          &assessor (FC)     &assessor (FC)    & 100000   \\
  nnrHills2nnr     &assessor (FC)     &assessor (FC)    & 100000   \\
  G328             &assessor (FC)     &assessor (FC)    & 100000   \\
  miaBasin2mc      &assessor (FC)     &assessor (FC)    & 100000   \\
  G402             &wmcap             & 0               & 100000   \\
  holeyland2lmiami &wmcap             & 0               & 100000   \\
  G136             &assessor (FC)     &assessor (FC)    & 100000   \\
  G406             &assessor (FC)     &assessor (FC)    & 100000   \\
  G343SUM          &assessor (FC)     &assessor (FC)    & 100000   \\
  usso             &assessor (FC)     &assessor (FC)    & 100000   \\
  G600             &assessor (FC)     &assessor (FC)    & 100000   \\
  G311             &wmcap             & 0               & 100000   \\
  G302             &wmcap             & 0               & 100000   \\
  G300G301         &wmcap             & 0               & 100000   \\
  S362             &wmcap             & 0               & 100000   \\
  G251G310         &wmcap             & 0               & 100000   \\
  sta2in           &wmcap             & 0               & 100000   \\
  G338             &wmcap             & 0               & 100000   \\
  G339             &wmcap             & 0               & 100000   \\
  G335             &wmcap             & 0               & 100000   \\
  G372HL           &wmcap             & 0               & assessor \\
  g380             &wmcap             & 0               & 100000   \\
  sta34out         &wmcap             & 0               & assessor \\
  G344             &wmcap             & 0               & assessor \\
  sta6out          &wmcap             & 0               & assessor \\
  G300G301bf       &wmcap             & 0               & 100000   \\
  \hline                                                                            
  \end{tabular}
 \end{center}
\end{table}
\normalsize

\begin{table}[!htb]
 \begin{center}
  \footnotesize
  \caption{Example of the LP constant structure parameter values (Part 1). }\label{strParaConst_1}
  \begin{tabular}{p{2.5cm}p{2.5cm}p{2.5cm}p{1.8cm}p{1.5cm}p{1.5cm}}\\[0.8ex]
                   &upstream       &downstream        &design         &use     &         \\
   structure       &basin          &basin             & capacity      &manCon  & w\_flow \\
  \hline
  C10A             &lo              &L8               &20000.0        &1       &0.1 \\
  S352             &lo              &L10L12           &20000.0        &1       &0.1 \\
  S351             &lo              &nnrHills         &20000.0        &1       &0.1 \\
  S354             &lo              &umiami           &20000.0        &1       &0.05\\ 
  C10              &lo              &ebwcd            &20000.0        &1       &0   \\
  C12              &lo              &eswcd            &20000.0        &1       &0   \\
  C12A             &lo              &closter          &20000.0        &1       &0   \\
  C4A              &lo              &ssdd             &20000.0        &1       &0   \\
  S236             &lo              &sfcd             &20000.0        &1       &0   \\
  S5AE             &L8              &C51W             &20000.0        &1       &0.1 \\
  L8canal2basin    &L8              &L8basin          &20000.0        &1       &0   \\
  CWPB2            &L8              &dummyLake        &20000.0        &1       &0   \\
  S155A            &C51W            &C51              &20000.0        &1       &0.1 \\
  C51Wcanal2basin  &C51W            &C51WBasin        &20000.0        &1       &0   \\
  S319             &C51W            &sta1e            &20000.0        &1       &0   \\
  C51canal2basin   &C51             &C51Basin         &20000.0        &1       &0   \\
  S155             &C51             &dummyLake        &20000.0        &1       &0   \\
  S5A              &L10L12          &STA1InflowCanal  &20000.0        &1       &0   \\
  L10L12toS5ABasin &L10L12          &S5ABasin         &20000.0        &1       &0   \\
  S2               &nnrHills        &lo               &20000.0        &1       &0   \\
  G371             &nnrHills        &L5E              &20000.0        &1       &5.0 \\
  nnr2nnrHills     &nnrHills        &NnrHillsBasin    &20000.0        &1       &0   \\
  S6               &nnrHills        &STA2InflowCanal  &20000.0        &1       &0.2 \\
  G370             &nnrHills        &sta34            &20000.0        &1       &0.1 \\
  S3               &umiami          &lo               &20000.0        &1       &0   \\
  G373             &umiami          &lmiami           &20000.0        &1       &5.0 \\
  mc2miaBasin      &umiami          &miamiBasin       &20000.0        &1       &0   \\
  G372             &umiami          &sta34sc          &20000.0        &1       &0.1 \\
  G410             &lmiami          &rotenberger      &20000.0        &1       &0.1 \\
  L4toBigCypRes    &lmiami          &bigCypRes        &20000.0        &1       &0   \\
  G507             &lmiami          &sta5             &20000.0        &1       &0.1 \\
  s8               &lmiami          &wca3a            &20000.0        &1       &0   \\
  L4gap            &lmiami          &dummyLake        &20000.0        &1       &0.001\\
  S7               &L5E             &wca2a            &20000.0        &1       &0   \\
  S150             &L5E             &wca3a            &20000.0        &1       &0   \\
  G407             &L3Canal         &lmiami           &20000.0        &1       &0   \\
  G353             &L3Canal         &sta6             &20000.0        &1       &0.1 \\
  \hline                                                                            
  \end{tabular}
 \end{center}
\end{table}
\normalsize

\begin{table}[!htb]
 \begin{center}
  \footnotesize
  \caption{Example of the LP constant structure parameter values (Part 2). }\label{strParaConst_2}
  \begin{tabular}{p{2.8cm}p{2.5cm}p{2.5cm}p{1.8cm}p{1.5cm}p{1.5cm}}\\[0.8ex]
                   &upstream       &downstream        &design         &use     &         \\
   structure       &basin          &basin             & capacity      &manCon  & w\_flow \\
  \hline
  basin2L8canal    &L8basin         &L8               &20000.0        &1       &0   \\
  L8DemandNode     &L8basin         &dummyLake        &20000.0        &1       &0   \\
  basin2C51Wcanal  &C51WBasin       &C51W             &20000.0        &1       &0   \\
  C51WDemandNode   &C51WBasin       &dummyLake        &20000.0        &1       &0   \\
  basin2C51canal   &C51Basin        &C51              &20000.0        &1       &0   \\
  C51DemandNode    &C51Basin        &dummyLake        &20000.0        &1       &0   \\
  S5ABasin2L10L12  &S5ABasin        &L10L12           &20000.0        &1       &0   \\
  S5ABasin2hills   &S5ABasin        &nnrHills         &20000.0        &1       &0   \\
  C10bp            &ebwcd           &lo               &20000.0        &1       &0   \\
  ebps             &ebwcd           &L10L12           &20000.0        &1       &0   \\
  C12bp            &eswcd           &lo               &20000.0        &1       &0   \\
  esps             &eswcd           &nnrHills         &20000.0        &1       &0   \\
  C12Abp           &closter         &lo               &20000.0        &1       &0   \\
  cfps             &closter         &nnrHills         &20000.0        &1       &0   \\
  C4Abp            &ssdd            &lo               &20000.0        &1       &0   \\
  sbpsrips         &ssdd            &nnrHills         &20000.0        &1       &0   \\
  sswestps         &ssdd            &umiami           &20000.0        &1       &0   \\
  S236bp           &sfcd            &lo               &20000.0        &1       &0   \\
  sfd5eps          &sfcd            &umiami           &20000.0        &1       &0   \\
  nnrHills2nnr     &NnrHillsBasin   &nnrHills         &20000.0        &1       &0   \\
  G328             &NnrHillsBasin   &STA2InflowCanal  &20000.0        &1       &0   \\
  miaBasin2mc      &miamiBasin      &umiami           &20000.0        &1       &0   \\
  G402             &rotenberger     &lmiami           &20000.0        &1       &0.1 \\
  holeyland2lmiami &holeyland       &lmiami           &20000.0        &1       &0.1 \\
  G136             &L1EBasin        &umiami           &20000.0        &1       &0   \\
  G406             &L2CBasin        &L3Canal          &20000.0        &1       &0.1 \\
  G343SUM          &L2CBasin        &sta5             &20000.0        &1       &0   \\
  usso             &annexBasin      &sta6             &20000.0        &1       &0   \\
  G600             &ussugarUnit2    &sta6             &20000.0        &1       &0   \\
  G311             &STA1InflowCanal &sta1e            & 1550.0        &1       &0.1 \\
  G302             &STA1InflowCanal &sta1w            & 3250.0        &1       &0   \\
  G300G301         &STA1InflowCanal &wca1             &20000.0        &1       &5.0 \\ 
  S362             &sta1e           &wca1             & 4200.0        &1       &0   \\
  G251G310         &sta1w           &wca1             & 3490.0        &1       &0   \\
  sta2in           &STA2InflowCanal &sta2             &20000.0        &1       &0   \\
  G338             &STA2InflowCanal &wca1             &20000.0        &1       &5.0 \\
  G339             &STA2InflowCanal &wca2a            &20000.0        &1       &5.0 \\
  G335             &sta2            &wca2a            & 3040.0        &1       &0   \\
  G372HL           &sta34sc         &holeyland        &20000.0        &1       &5.0 \\
  g380             &sta34sc         &sta34            &20000.0        &1       &0   \\
  sta34out         &sta34           &lmiami           & 5844.0        &1       &0   \\
  G344             &sta5            &lmiami           & 1800.0        &1       &0.1 \\
  sta6out          &sta6            &lmiami           & 1197.0        &1       &0   \\
  G300G301bf       &wca1            &STA1InflowCanal  &20000.0        &1       &5.0 \\     
  \hline                                                                            
  \end{tabular}
 \end{center}
\end{table}
\normalsize

