\chapter{Assessor Coordinator}

The Management Simulation Engine (MSE) implements methods for a wide
range of management policies and operational controls for structures
simulated in the HSE.  Many of these methods have been developed and
used successfully in legacy models, such as the South Florida Water
Management Model (SFWMD, 2005) \nocite{sfwmm:2005}, South Florida
Regional Routing Model (Trimble, 1986) \nocite{trimble:86} and the
Upper Kissimmee Chain of Lakes Model (Fan, 1986) \nocite{fan:86}.  The
\emph{assessor coordinator} is used to implement these legacy methods
in the MSE.

The assessor coordinator supervises a set of WCU assessors that are
designed to quantify water supply and flood control needs, and specify
control structure releases in the MSE network. Water control units are
HSE water bodies that are managed as discrete entities in the MSE
Nework. A water control unit (WCU) can be a lake, basin, canal reach
(assembly of canal segments), or mesh (assembly of cells). The
regional water supply and flood control needs are met by routing water
through the MSE Network with coordinated releases through the WCU's
inlet and outlet control structures. A control structure is an HSE
water mover paired with a controller and it is represented in the MSE
Network by an MSE Node. The controller serves as the interface between
the HSE and MSE.

The WCU assessors follow these steps to set control structure releases
at MSE Nodes:

\begin{enumerate}
 \item Quantify water supply and flood control needs for the WCU;
 \item Distribute water supply and flood control needs to the WCU
   inlet and outlet MSE Nodes;
 \item Limit MSE Node releases based infrastructure and policy
   constraints; and
 \item Impose MSE Node release on the corresponding HSE water mover
   through the controller.
 \end{enumerate}

Water supply and flood control releases are subject to infrastructure
and policy constraints. Infrastructure constraints include water
mover capacity, design capacity and limited storage capacity upstream
or downstream of the control structure.  Policy constraints reflect
local and regional management strategies for optimizing system
performance, resolving conflicts between competing uses, and achieving
specified management objectives.

Water management policies in the assessor coordinator can be
implemented in a variety of ways.  A rule or policy that is regional
in scope and spans multiple WCUs can be simulated using decision trees
(see Chapter \ref {chapter:RegionalManager}) using a management
constraint (``manCon'').  Management constraints are assigned to each
MSE node and can used to selectively limit flood control and/or water
supply releases based on landscape type.  Examples include the
regulation schedule for Lake Okeechobee, where the flood control
release through an outlet is based on the water level in the lake,
degree of flooding in the EAA, and water levels in the Water
Conservation Areas.  Other policies are narrower in scope and can be
implemented generically at the WCU level.  For example, a policy
dictates how runoff from a basin is to be distributed across multiple
outlets. Some policies are unique to a WCU and are simulated using
customized source code in specialized assessors. For example, the
pulse release assessor is used to compute the flood control release
from Lake Okeechobee to the estuary, based on a pulse release
hydrograph.  Output from a special assessor can be monitored and used
as input for the assessor coordinator.

\section{WCU Assessors}
\label{chapter:WaterControlUnitManagement}

Water control units are managed by WCU assessors to meet user defined
flood control / water supply objectives.  WCU assessors are designed
to quantify the water supply and flood control needs for the WCU and
supervise the operational response to these needs at the WCU's inlets
and outlets.  A variety of WCU assessors have been developed to
simulate the needs of different landscapes (e.g., natural,
agricultural and urban) and implement decision making processes that
manage the control structures.

Since the simulated inlet and outlet releases from a WCU affect the
state of the upstream and downstream WCUs, the order in which WCU
assessors are processed is important.  The assessor coordinator
manages the sequencing by placing the WCU assessors in water supply
and flood control queues and executing their respective Assess()
methods in the order specified by the user. 

\begin{itemize}
 \item Water supply assessors \-- Queued from downstream to upstream.
   Each assessor quantifies the water supply requirement, which
   includes the local WCU demand, regional demand from downstream
   water supply outlets, and boundary conditions.  The net water
   supply requirement for the WCU is distributed to the water supply inlets.
   The regional water supply requirements are propagated up the MSE Network
   by repeating this process for the remaining upstream water supply
   assessors in the queue.

 \item Flood control assessors \-- Queued from upstream to
   downstream. Each assessor quantifies the flood control
   requirements, which includes the local WCU excess and regional
   contributions from the inlets, outlets, and boundary
   conditions. The outlet releases for the WCU are set by the
   assessor, considering the excess in the WCU, downstream water
   supply needs, downstream storage capacity and operational criteria
   at the outlets. All releases are subject to physical and management
   constraints. The water supply and flood control requirements for
   the MSE Network are resolved for the WCU by setting outlet releases
   and repeating this process for the remaining downstream flood
   control assessors in the queue.
\end{itemize}

WCU assessors have default water supply and flood control methods to
assess rquirements in the water control unit and coordinate
operational responses at the respective inlets and outlets.
Alternative methods are available through ``packages'', where the
default behavior is modified to simulate the specialized management
and operational control for a particular WCU.  The default and
alternative methods for water supply and flood control assessors are
described in the following sections.

